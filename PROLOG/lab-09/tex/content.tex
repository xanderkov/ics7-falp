\chapter{Практические задания}


\begin{enumerate}
	\item Создать базу знаний «Предки», позволяющую наиболее эффективным способом
	(за меньшее количество шагов, что обеспечивается меньшим количеством
	предложений БЗ - правил), и используя разные варианты (примеры) простого вопроса,
	(указать: какой вопрос для какого варианта) определить:
	\begin{enumerate}
		\item по имени субъекта определить всех его бабушек (предки 2-го колена),
		\item по имени субъекта определить всех его дедушек (предки 2-го колена),
		\item по имени субъекта определить всех его бабушек и дедушек (предки 2-го колена),
		\item по имени субъекта определить его бабушку по материнской линии (предки 2-го колена),
		\item по имени субъекта определить его бабушку и дедушку по материнской линии
		(предки 2-го колена).
	\end{enumerate}
	Минимизировать количество правил и количество вариантов вопросов. Использовать
конъюнктивные правила и простой вопрос. Для одного из вариантов ВОПРОСА задания 1
составить таблицу, отражающую конкретный порядок работы системы

	\item Дополнить базу знаний правилами, позволяющими найти
	\begin{enumerate}
		\item Максимум из двух чисел: без использования отсечения, с использованием отсечения
		\item Максимум из трех чисел: без использования отсечения, с использованием отсечения
	\end{enumerate}

	Убедиться в правильности результатов.
	Для каждого случая пункта 2 обосновать необходимость всех условий тела.
	Для одного из вариантов ВОПРОСА и каждого варианта задания 2 составить
	таблицу, отражающую конкретный порядок работы системы.

\end{enumerate}


\begin{lstlisting}
	parent(lorem, ipsum, f).
	parent(dolor, ipsum, m).

	parent(ipsum, sit, m).
	parent(labore, sit, f).

	parent(ipsum, amet, m).
	parent(labore, amet, f).

	parent(consectetur, adipiscing, m).
	parent(elit, adipiscing, f).

	parent(sit, do, m).
	parent(sed, do, f).

	parent(adipiscing, eiusmod, m).
	parent(amet, eiusmod, f).
	parent(adipiscing, tempor, m).
	parent(amet, tempor, f).
	parent(adipiscing, incididunt, m).
	parent(amet, incididunt, f).

	grandparent(X, Y)   :- parent(X, Z, _), parent(Z, Y, _).
	grandmother(X, Y)   :- parent(X, Z, f), parent(Z, Y, _).
	grandfather(X, Y)   :- parent(X, Z, m), parent(Z, Y, _).

	maternal_grandmother(X, Y)   :- parent(X, Z, f), parent(Z, Y, f).
	maternal_grandparent(X, Y)   :- parent(X, Z, _), parent(Z, Y, f).
	
\end{lstlisting}


\begin{lstlisting}
	max2(A, B, B)    :- B >= A.
	max2(A, B, A)    :- A >= B.
	max3(A, B, C, A)    :- A >= B, A >= C.
	max3(A, B, C, B)    :- B >= A, B >= C.
	max3(A, B, C, C)    :- C >= A, C >= B.
	max2clip(A, B, B)    :- B >= A, !.
	max2clip(A, _, A).
	max3clip(A, B, C, A)    :- A >= B, A >= C, !.
	max3clip(_, B, C, B)    :- B >= C, !.
	max3clip(_, _, C, C).
	
\end{lstlisting}

\begin{table}[ht!]
	\centering
	\caption{Задание 1}
	\label{decisions}
	\begin{tabular}{|p{0.3cm}|p{4cm}|p{7.5cm}|p{4cm}|}
			\hline
			\textbf{N} & \textbf{Состояние резольвенты} & \textbf{Для каких теромв запускатеся алгоритм унификации и результат подстановки} & \textbf{Дальнейшие действия: прямой ход или откат}\\
			
			\hline

			1 & grandparent(N, incididunt)
			& Запуск алгоритма унификации для 
			grandparent(N, incididunt) и parent(lorem, ipsum, f)
			Унификация неуспешна.
			Подстановка: Пусто
			& Прямой ход, переход к следующему
			предложению \\

			\hline

			20 & parent(X, Z, $\_$), parent(Z, Y, $\_$)
			& Запуск алгоритма унификации для 
			grandparent(N, incididunt) и grandparent(X, Y) 
			Унификация успешна.
			Подстановка: (X = N, Z = incididunt)
			& Прямой ход, переход к следующему
			предложению \\

			\hline

			20 & parent(X, Z, $\_$), parent(Z, Y, $\_$)
			& Запуск алгоритма унификации для 
			parent(N, incididunt, $\_$) и parent(lorem, ipsum, f)
			Унификация неуспешна.
			Подстановка: (X = N, Z = incididunt)
			& Прямой ход, переход к следующему
			предложению \\

			\hline

			45 & Пустая
			& Запуск алгоритма унификации для 
			parent(N, incididunt, $\_$) и parent(adipiscing, incididunt, m)
			Унификация успешна.
			Подстановка: (X = N, Z = incididunt, N = adipiscing)
			& Получен результат, Откат
			предложению \\

			\hline

			50 & Пустая
			& Запуск алгоритма унификации для 
			parent(N, adipiscing, $\_$) и parent(consectetur, adipiscing, m)
			Унификация успешна.
			Подстановка: (X = N, Z = incididunt, N = consectetur)
			& Получен результат, Откат
			предложению \\

			\hline
		
	\end{tabular}
\end{table}


\begin{table}[ht!]
	\centering
	\caption{Максимум из трех без использования отсечения}
	\label{decisions}
	\begin{tabular}{|p{0.3cm}|p{4cm}|p{7.5cm}|p{4cm}|}
			\hline
			\textbf{N} & \textbf{Состояние резольвенты} & \textbf{Для каких теромв запускатеся алгоритм унификации и результат подстановки} & \textbf{Дальнейшие действия: прямой ход или откат}\\
			\hline
			1 & max3(3, 1, 2, Max) & Запуск алгоритма унификации для max3(3, 1, 2, Max) и max2(A, B, A). Унификация неуспешна. Подстановка: Пусто & Пряомй ход, переход к следующему терму (Резольвента не пуста) \\
			\hline

			3 & 3 >= 2, 3 >= 1
			& Запуск алгоритма унификации для max3(3, 1, 2, Max)
			и max3(A, B, C, A).
			Унификация успешна. Подстановка: (A = 3, B = 1, C = 2, Max = A)
			& Прямой ход, решение цели
			резольвенты 3 >= 2 \\

			\hline

			4 & 3 >= 1
			& 3 >= 2 Верно. Подстановка: (A = 3, B = 1, C = 2, Max = A)
			& Прямой ход, решение цели
			резольвенты 3 >= 1\\

			\hline

			5 & Пусто
			& 3 >= 1 Верно. Подстановка: (A = 3, B = 1, C = 2, Max = A)
			& Переменная Max реконкретизирована.
			Откат, переход к следующему относительно шага 3
			предложению.\\

			\hline

			6 & 1 >= 2, 1 >= 3
			& Запуск алгоритма унификации для max3(3, 1, 2, Max) и
			max3(A, B, C, B).
			Унификация успешна. Подстановка: (A = 3, B = 1, C = 2, Max = B)
			& Прямой ход, решение цели
			резольвенты 1 >= 2\\
		
			\hline

			7 & max3(3, 1, 2, Max)
			& 1 >= 2 Неверно. Подстановка: Пусто
			& Откат, переход к следующему 
			предложению относительно шага 6.\\
		
			\hline

			8 & 2 >= 1, 2 >= 3
			& Запуск алгоритма унификации для Max3(3, 1, 2, Max) и
			Max3(Num1, Num2, Num3, Num3).
			Унификация успешна Подстановка: (A = 3, B = 1, C = 2, Max = C)
			& Прямой ход, решение цели
			резольвенты 2 >= 1.\\
		
			\hline

			9 & 2 >= 3
			& 2 >= 1 Верно. Подстановка: (A = 3, B = 1, C = 2, Max = C)
			& Прямой ход, решение цели
			резольвенты 2 >=3.\\
		
			


	\end{tabular}
\end{table}

\begin{table}[ht!]
	\centering
	\caption{Максимум из трех без использования отсечения}
	\label{decisions}
	\begin{tabular}{|p{0.3cm}|p{4cm}|p{7.5cm}|p{4cm}|}

			\hline
			10 & max3(3, 1, 2, Max)
			& 2 >= 3 Неверно. Подстановка: Пусто
			& Откат, переход к след. предл. относит. шага 8.\\
		
			\hline

			11 & max3(3, 1, 2, Max)
			& Запуск алгоритма унификации для max3(3, 1, 2, Max) и
			max3clip(A, B, C, A).
			Унификация неуспешна. Подстановка: Пусто
			& Прямой ход, переход к следующему предложению.\\
		
			\hline

			13 & max3(3, 1, 2, Max)
			& Запуск алгоритма унификации для Max3(3, 1, 2, Max) 
			и max3clip($\_$, $\_$, Num3, Num3). Унификация неуспешна. Подстановка: Пусто
			& Завершение работы, вывод результата на экран.\\
		
			\hline
	\end{tabular}
\end{table}


\begin{table}[ht!]
	\centering
	\caption{Максимум из трех с использования отсечения}
	\label{decisions}
	\begin{tabular}{|p{0.3cm}|p{4cm}|p{7.5cm}|p{4cm}|}
			\hline
			\textbf{N} & \textbf{Состояние резольвенты} & \textbf{Для каких теромв запускатеся алгоритм унификации и результат подстановки} & \textbf{Дальнейшие действия: прямой ход или откат}\\
			\hline
			1 & max3clip(3, 1, 2, Max) П 
			&  Запуск алгоритма унификации 
			для max3clip(3, 1, 2, Max) и
			max2(A, B, A).
			Унификация неуспешна.
			Подстановка: Пусто 
			& Прямой ход, переход к следующему предложению. \\
			\hline

			8 & 3 >= 2, 3 >= 1, !
			& Запуск алгоритма унификации для
			max3clip(3, 1, 2, Max) и
			max3clip(A, B, C, A).
			Унификация успешна. Подстановка: (A=3, B=1, C=2, Max=A)
			& Прямой ход, решение цели из
			резольвенты 3 >= 2\\
		
			\hline

			9 & max3(3, 1, 2, Max)
			& 3 >= 2 Верно. Подстановка: (A=3, B=1, C=2, Max=B)
			& Прямой ход, решение цели из
			резольвенты 3 >= 1. \\
		
			\hline

			10 & Пусто
			& 3 >= 1 Верно. Подстановка: (A=3, B=1, C=2, Max=3)
			& Реконкретизация Max, оператор
			отсечения, откат к пункту 8, завершение работы, поскольку
			метка на последнем предложении.\\
		
			\hline
	\end{tabular}
\end{table}