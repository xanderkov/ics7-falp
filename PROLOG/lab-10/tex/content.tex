\chapter{Практические задания}

\textbf{Цель работы}: изучить рекурсивные способы организации программ на Prolog,
методы формирования эффективных рекурсивных программ и порядок реализации таких программ.

\textbf{Задачи работы}: приобрести навыки использования рекурсии на Prolog, эффективного
способа ее организации и прядка работы соответствующей программы.
Изучить возможность и необходимость использования системных предикатов в рекурсивной
программе на Prolog, принципы и особенности порядка работы такой программы. Способ
формирования и изменения резольвенты в этом случае и порядок формирования ответа.

\section*{Задание}

Используя хвостовую рекурсию, разработать программу, позволяющую найти

\begin{enumerate}[label=\arabic*.]
	\item n!,
	\item n-е число Фибоначчи.
\end{enumerate}

Убедиться в правильности результатов.

Для одного из вариантов ВОПРОСА и каждого задания составить таблицу, отражающую
конкретный порядок работы системы:

Т.к. резольвента хранится в виде стека, то состояние резольвенты требуется отображать в
столбик: вершина – сверху! Новый шаг надо начинать с нового состояния резольвенты!

Для одного из вариантов ВОПРОСА составить таблицу, отражающую конкретный порядок работы системы.

\pagebreak

\section*{Ход работы}

\begin{center}
	\begin{lstlisting}[caption=Описательные разделы программы]
	factorial(1, Result, X):- Result=X, !.
	factorial(N, Result, X):-
		Xnew=N*X,
		Nm=N-1,
		factorial(Nm, Result, Xnew).


	fibonacci(1, Result, Last1, Last2):- Result=0, !.
	fibonacci(2, Result, Last1, Last2):- Result=Last2, !.
	fibonacci(N, Result, Last1, Last2):-
		Last2new=Last1+Last2,
		Nnew=N-1,
		fibonacci(Nnew, Result, Last2, Last2new).
	\end{lstlisting}
\end{center}

\noindent{Решение:} 
\begin{enumerate}
	\item[1.] factorial(10, F, 1).
	\item[2.] fibonacci(22, F, 0, 1).
\end{enumerate}

\chapter*{Приложение А}

\begin{table}[!h]
	\begin{center}
		\begin{threeparttable}
			\captionsetup{justification=raggedright,singlelinecheck=off}
			\caption{Порядок работы системы для factorial(3, F, 1)}
			\begin{tabular}{|m{0.5cm}|m{3.5cm}|m{8.5cm}|m{3cm}|}
				\hline
				№ & Резольвента &Термы, подстановка & продолжение \\
				\hline
				1 & factorial(3, F, 1) & T1=factorial(3, F, 1) \newline T2=factorial(1, Result, X) нет & следующее правило \\
				\hline
				2 & factorial(3, F, 1) & T1=factorial(3, F, 1) \newline T2=factorial(N, Result, X) да \newline $\theta$ = \{N=3, Result=F, X=1\} & переход к \newline телу правила \\
				\hline
				3 & Xnew=3*1, \newline Nm=3-1, \newline factorial(Nm, Result, Xnew) & Xnew=3*1 да \newline $\theta$ = \{Xnew=3*1=3\} & следующее правило \\
				\hline
				4 & Nm=N-1, \newline factorial(Nm, Result, 3) & Nm=N-1 \newline да \newline $\theta$ = \{Nm=3-1=2\} & следующее правило \\
				\hline
				5 & factorial(3, Result, 3) & T1=factorial(3, Result, 3) \newline T2=factorial(1, Result, X) нет & следующее правило \\
				\hline
				6 & factorial(2, Result, 3) & T1=factorial(3, Result, 3) \newline T2=factorial(N, Result, X)\newline да \newline $\theta$ = \{N=2, Result=Result, X=3\} & переход к \newline телу правила \\
				\hline
				7 & Xnew=2*3, \newline Nm=2-1, \newline factorial(Nm, Result, Xnew) & Xnew=2*3 \newline да \newline $\theta$ = \{Xnew=2*3=6\} & следующее правило \\
				\hline
				8 & Nm=2-1, \newline factorial(Nm, Result, 6) & Nm=2-1 \newline да \newline $\theta$ = \{Nm=2-1=1\} & следующее правило \\
				\hline
				9 & factorial(1, Result, 6) & T1=factorial(1, Result, 6) \newline T2=factorial(1, Result, X) \newline да \newline $\theta$ = \{Result=Result, X=6\} & переход к \newline телу правила \\
				\hline
				10 & Result=6, ! & Result=6 \newline да \newline $\theta$ = \{Result=6\} & следующее правило \\
				\hline
				11 & ! & ! \newline остановка обработки процедуры & откат \\
				\hline
			\end{tabular}
\end{threeparttable}
\end{center}
\end{table}				
				
\begin{table}[h]
	\begin{center}
		\begin{threeparttable}
			\captionsetup{justification=raggedright,singlelinecheck=off}
			\caption{Порядок работы системы для fibonacci(4, F, 0, 1)}
			\begin{tabular}{|m{0.5cm}|m{3.5cm}|m{8.5cm}|m{3cm}|}
				\hline
				№ & Резольвента &Термы, подстановка & продолжение \\
				\hline
				1 & fibonacci(4, F, 0, 1) & T1=fibonacci(4, F, 0, 1) \newline T2=fibonacci(1, Result, Last1, Last2) \newline нет & следующее правило \\
				\hline
				2 & fibonacci(4, F, 0, 1) & T1=fibonacci(4, F, 0, 1) \newline T2=fibonacci(2, Result, Last1, Last2) \newline нет & следующее правило \\
				\hline
				3 & fibonacci(4, F, 0, 1) & T1=fibonacci(4, F, 0, 1) \newline T2=fibonacci(N, Result, Last1, Last2)\newline да \newline $\theta$ = \{N=4, Result=F, Last1=0, Last2=1\} & переход к \newline телу правила \\
				\hline
				4 & Last2new=0+1, \newline Nnew=4-1, \newline fibonacci(Nnew, Result, 1, Last2new). & Last2new=0+1 \newline да \newline $\theta$ = \{Last2new=0+1=1\} & следующее правило \\
				\hline
				5 & Nnew=4-1, \newline fibonacci(Nnew, Result, 1, 1) & Nm=4-1 \newline да \newline $\theta$ = \{Nm=4-1=3\} & следующее правило \\
				\hline
				6 & fibonacci(3, Result, 1, 1) & T1=fibonacci(3, Result, 1, 1) \newline T2=fibonacci(1, Result, Last1, Last2) \newline нет & следующее правило \\
				\hline
				7 & fibonacci(3, Result, 1, 1) & T1=fibonacci(4, Result, 0, 1) \newline T2=fibonacci(2, Result, Last1, Last2) \newline нет & следующее правило \\
				\hline
				8 & fibonacci(3, Result, 1, 1) & T1=fibonacci(3, Result, 1, 1) \newline T2=fibonacci(N, Result, Last1, Last2)\newline да \newline $\theta$ = \{N=3, Result=Result, Last1=1, Last2=1\} & переход к \newline телу правила \\
				\hline
				9 & Last2new=1+1, \newline Nnew=3-1, \newline fibonacci(Nnew, Result, 1, Last2new). & Last2new=1+1 \newline да \newline $\theta$ = \{Last2new=1+1=2\} & следующее правило \\
				\hline
				10 & Nnew=3-1, \newline fibonacci(Nnew, Result, 1, 2) & Nm=3-1 \newline да \newline $\theta$ = \{Nm=3-1=2\} & следующее правило \\
				\hline
			\end{tabular}
\end{threeparttable}
\end{center}
\end{table}
	

\begin{table}[h]
	\begin{center}
		\begin{threeparttable}
			\captionsetup{justification=raggedright,singlelinecheck=off}
			\caption{Порядок работы системы для fibonacci(4, F, 0, 1)}
			\begin{tabular}{|m{0.5cm}|m{3.5cm}|m{8.5cm}|m{3cm}|}
				\hline
				№ & Резольвента &Термы, подстановка & продолжение \\
				\hline
				11 & fibonacci(2, Result, 1, 2) & T1=fibonacci(2, Result, 1, 2) \newline T2=fibonacci(1, Result, Last1, Last2) \newline нет & следующее правило \\
				\hline
				12 & fibonacci(2, Result, 1, 2) & T1=fibonacci(2, Result, 1, 2) \newline T2=fibonacci(2, Result, Last1, Last2) \newline да \newline $\theta$ = \{Result=Result, Last1=1, Last2=2\} & переход к \newline телу правила \\
				\hline
				13 & Result=2, \newline ! & Result=2 \newline да \newline $\theta$ = \{Result=2\} & следующее правило \\
				\hline
				14 & ! & ! \newline остановка обработки процедуры & откат \\
				\hline
			\end{tabular}
		\end{threeparttable}
	\end{center}
\end{table}
