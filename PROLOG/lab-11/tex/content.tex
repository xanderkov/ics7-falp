\chapter*{Задание}

Используя хвостовую рекурсию, разработать (комментируя назначение аргументов)
эффективную программу, позволяющую:

\begin{enumerate}
	\item Найти длину списка (по верхнему уровню);
	\item Найти сумму элементов числового списка;
	\item Найти сумму элементов числового списка, стоящих на нечетных
	      позициях исходного
	      списка (нумерация от 0);
	\item Сформировать список из элементов числового списка, больших
	      заданного значения;
	\item Удалить заданный элемент из списка (один или все вхождения).
	\item Объединить два списка.
\end{enumerate}

\begin{lstlisting}
	list_length(List, Length) :- list_length(List, 0, Length).
	list_length([], Acc, Acc) :- !.
	list_length([_|T], Acc, Length) :- !,
		NewAcc is Acc + 1,
		list_length(T, NewAcc, Length).

	sum_list(List, Sum) :- sum_list(List, 0, Sum), !.
	sum_list([], Acc, Acc) :- !.
	sum_list([H|T], Acc, Sum) :- 
		NewAcc is Acc + H,
		sum_list(T, NewAcc, Sum).

	sum_even_positions(List, Sum) :- !,
		sum_even_positions(List, 0, Sum).

	sum_even_positions([], Sum, Sum) :- !.
	sum_even_positions([_], Sum, Sum).
	sum_even_positions([_, X|T], Acc, Sum) :-
		NewAcc is Acc + X,
		sum_even_positions(T, NewAcc, Sum).

	filter_gt_y([], _, []) :- !.
	filter_gt_y([X|Xs], Y, [X|Ys]) :- X > Y, !, filter_gt_y(Xs, Y, Ys).
	filter_gt_y([X|Xs], Y, Ys) :- X =< Y, !, filter_gt_y(Xs, Y, Ys).

	delete_elem_once(_, [], []) :- !.
	delete_elem_once(X, [X|Tail], Tail) :- !.
	delete_elem_once(X, [Y|Tail], [Y|Tail1]) :-
		delete_elem_once(X, Tail, Tail1).

	delete_all(_, [], []) :- !.
	delete_all(X, [X|Tail], Result) :- !,
		delete_all(X, Tail, Result).
	delete_all(X, [Y|Tail], [Y|Tail1]) :- delete_all(X, Tail, Tail1).

	concat([], L, L) :- !.
	concat([H|T], L, [H|Result]) :- concat(T, L, Result).
\end{lstlisting}

\begin{table}[!ht]
	\begin{center}
		\begin{threeparttable}

			\captionsetup{justification=raggedright,singlelinecheck=off}
			\caption{Порядок работы системы для
				list\_length([1,2,3], Res)}
			\begin{tabular}{|m{0.5cm}|m{3.5cm}|m{8.5cm}|m{3cm}|}
				\hline
				\textbf{N}                                                                 & \textbf{Состояние резольвенты}                                                & \textbf{Для каких термов
				запускается алгоритм унификации и результат подстановки}                   & \textbf{Дальнейшие
				действия}                                                                                                                                                                                          \\
				\hline
				1                                                                          & list\_length([1,2,3], Res).                                                   & list\_length([1,2,3], 0). Результат:
				Унификация прошла успешно, переменная Acc принимает значение 0.            & Прямой ход.
				\\
				\hline
				2                                                                          & list\_length([1,2,3], Acc, Res), \newline Res = Length, \newline Acc = 0.                       &
				list\_length([1,2,3], 0). Результат: Унификация прошла успешно, переменная Acc
				принимает значение 0.                                                      & Прямой ход.                                                                                                           \\
				\hline
				2                                                                          & list\_length([2,3], 1, Res).                                                  & list\_length([2,3], 1, Length).
				Результат: Унификация прошла успешно, переменная Acc принимает значение 1. &
				Прямой ход.                                                                                                                                                                                        \\
				\hline
				3                                                                          & list\_length(List, Acc, Res), \newline Res=0,\newline Acc=1, \newline List=[2,3].
				                                                                           & list\_lengthh(List, Acc, Res)list\_lengthth([\_|T],Acc, Length). Результат:
				Унификация прошла успешно, переменная NewAcc принимает значение Acc + 1 = 2,
				List=T.                                                                    & Прямой ход.                                                                                                           \\
				\hline

				4                                                                          & list\_length(List, Acc, Res),\newline Res=0, \newline Acc=2, \newline List=[3].
				                                                                           & list\_lengthh(List, Acc, Res)list\_lengthth([\_|T],Acc, Length). Результат:
				Унификация прошла успешно, переменная NewAcc принимает значение Acc + 1 = 3,
				List=T.                                                                    & Прямой ход.                                                                                                           \\
				\hline
				5                                                                          & list\_length(List, Acc, Res), \newline Res=0, \newline Acc=3, \newline List=[]
				                                                                           & list\_lengthh([], Acc, Acc). Результат: Унификация прошла успешно, переменная
				Acc принимает значение 3.                                                  & Обратный ход.                                                                                                         \\
				\hline
				6                                                                          & list\_length(List, Acc, Res), \newline Res=3, \newline Acc=3, \newline List=[]
				                                                                           & list\_lengthh([], Acc, Acc). Результат: Унификация прошла успешно.            & Обратный
				ход.                                                                                                                                                                                               \\
				\hline
				7                                                                          & list\_length([1,2,3], Res).                                                   & list\_lengthh([1,2,3], 0). Результат:
				Унификация прошла успешно.                                                 & Обратный ход.                                                                                                         \\
				\hline
			\end{tabular}
		\end{threeparttable}
	\end{center}
\end{table}