\chapter{Практические задания}

\begin{enumerate}[wide=0pt]

	\item \textit{Напишите функцию, которая уменьшает на 10 все числа из списка-аргумента этой
	функции, проходя по верхнему уровню списковых ячеек. ( * Список смешанный
	структурированный)}

		\begin{lstlisting}
			(defun minus-ten-list (lst) (
				mapcar #'(
							lambda (elem) (
								cond 
									((numberp elem) (- elem 10))
									(T elem)
							)
						) lst
				)
			)
		\end{lstlisting}



	\item  \textit{Написать функцию которая получает как аргумент список чисел, а возвращает список
	квадратов этих чисел в том же порядке.}

	\begin{lstlisting}
		(defun square-lst (lst) (
			mapcar #'(
						lambda (elem) (
							cond 
								((numberp elem) (* elem elem))
								(T elem)
						)
					) lst
			)
		)

	\end{lstlisting}

	\item  \textit{Напишите функцию, которая умножает на заданное число-аргумент все числа из
	заданного списка-аргумента, когда
	a) все элементы списка --- числа,
	б) элементы списка -- любые объекты.}

	\begin{lstlisting}

		(defun multiply-lst-numbers (lst x) (
			mapcar #'(
						lambda (elem) (
							* elem x
						)
					) lst
			)
		)


		(defun multiply-lst (lst x) (
			mapcar #'(
						lambda (elem) (
							cond 
								((numberp elem) (* elem x))
								(T elem)
						)
					) lst
			)
		)


	\end{lstlisting}



	\item  \textit{Написать функцию, которая по своему списку-аргументу lst определяет является ли он
	палиндромом (то есть равны ли lst и (reverse lst)), для одноуровнего смешанного
	списка.}

	\begin{lstlisting}
		(defun polyp (lst) (
			cond ((eql 
				(find-if-not #'oddp

					(mapcar #'(
						lambda (elem revelem) (
							cond 
								((eql elem revelem) 1)
								(T 0)
						)
						) lst (reverse lst)
					)
				)
				Nil) '(T))
				(T Nil)
		)
		)
	\end{lstlisting}


	\item  \textit{Используя функционалы, написать предикат set-equal, который возвращает t, если два
	его множества-аргумента (одноуровневые списки) содержат одни и те же элементы,
	порядок которых не имеет значения.}
	
	\begin{lstlisting}
		(defun set-equalp (set1 set2)
		(
			and (eql (length set1) (length set2))
						(every #'(lambda (elem) 
									(member elem set2 :test #'equal)
								 ) set1)
						(every #'(lambda (elem) 
									(member elem set1 :test #'equal)
								 ) set2)
		)
		)
		
	\end{lstlisting}

	\item  \textit{Напишите функцию, select-between, которая из списка-аргумента, содержащего только
	числа, выбирает только те, которые расположены между двумя указанными числами -
	границами-аргументами и возвращает их в виде списка (упорядоченного по
	возрастанию (+ 2 балла)).}

	\begin{lstlisting}
		(defun select-between (lst l r)
		(
			sort (reduce #'(
							lambda (res el) (
								cond ((and (> el l) (< el r)) (cons el res))
										(T res)
				
							)
							)
			lst :initial-value ()) #'<
		)
		)

	\end{lstlisting}

	\item  \textit{Написать функцию, вычисляющую декартово произведение двух своих списковаргументов. ( Напомним, что А х В это множество всевозможных пар (a b), где а
	принадлежит А, принадлежит В.)}
	\begin{lstlisting}
		(defun combinations(lst1 lst2)
		(mapcan #'(lambda (inner)
					(mapcar #'(lambda (outer)
								(list outer inner)) 
							lst1)) 
				lst2))
	\end{lstlisting}

	\item  \textit{Почему так реализовано reduce, в чем причина?
	(reduce \#'+ ()) -> 0 (reduce \#'* ()) -> 1}
	
	\begin{lstlisting}
		(reduce #'+ ()) -> 0
		(reduce #'* ()) -> 1
	\end{lstlisting}

	начальное занчение функции + --- 0

	начальное значение функции * --- 1



	\item  \textit{* Пусть list-of-list список, состоящий из списков. Написать функцию, которая
	вычисляет сумму длин всех элементов list-of-list (количество атомов), т.е. например
	для аргумента
	 ((1 2) (3 4)) -> 4.}

	\begin{lstlisting}
		(defun list-of-list (lst)
			(apply #'+ (
						mapcar #'(
							lambda (elem) (
								cond ((listp elem) (list-of-list elem))
									(t 1)
							)
						)
						lst
					)
			)
		)
	\end{lstlisting}


\end{enumerate}