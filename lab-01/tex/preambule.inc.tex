\usepackage{cmap} % Улучшенный поиск русских слов в полученном pdf-файле
\usepackage[T2A]{fontenc} % Поддержка русских букв
\usepackage[utf8]{inputenc} % Кодировка utf8
\usepackage[english,russian]{babel} % Языки: русский, английский
%\usepackage{pscyr} % Нормальные шрифты
\usepackage{enumitem}


\usepackage[14pt]{extsizes}

\usepackage{graphicx}
\usepackage{multirow}

\usepackage{tikz}
\usepackage{pgfplots}
\usepackage{subcaption}
\usepackage{caption}
\captionsetup{labelsep=endash}
\captionsetup[figure]{name={Рисунок}}
\captionsetup[subtable]{labelformat=simple}
\captionsetup[subfigure]{labelformat=simple}
\renewcommand{\thesubtable}{\text{Таблица }\arabic{chapter}\text{.}\arabic{table}\text{.}\arabic{subtable}\text{ --}}
\renewcommand{\thesubfigure}{\text{Рисунок }\arabic{chapter}\text{.}\arabic{figure}\text{.}\arabic{subfigure}\text{ --}}

\usepackage{amsmath}

\usepackage{geometry}
\geometry{left=30mm}
\geometry{right=15mm}
\geometry{top=20mm}
\geometry{bottom=20mm}

\usepackage{titlesec}
\titleformat{\section}
{\normalsize\bfseries}
{\thesection}
{1em}{}
\titlespacing*{\chapter}{0pt}{-30pt}{8pt}
\titlespacing*{\section}{\parindent}{*4}{*4}
\titlespacing*{\subsection}{\parindent}{*4}{*4}

\usepackage{setspace}
\onehalfspacing % Полуторный интервал

\frenchspacing
\usepackage{indentfirst} % Красная строка после заголовка
\setlength\parindent{1.25cm}


\usepackage{titlesec}
\titleformat{\chapter}{\LARGE\bfseries}{\thechapter}{20pt}{\LARGE\bfseries}
\titleformat{\section}{\Large\bfseries}{\thesection}{20pt}{\Large\bfseries}


% Настройки оглавления
\usepackage{xcolor}
\usepackage{multirow}

\usepackage[pdftex]{hyperref} % Гиперссылки
\hypersetup{hidelinks}

% Листинги 
\usepackage{listings}

\definecolor{darkgray}{gray}{0.15}

\lstset{
	language=python, % выбор языка для подсветки
	basicstyle=\small\sffamily, % размер и начертание шрифта для подсветки кода
	numbers=left, % где поставить нумерацию строк (слева\справа)
	%numberstyle=, % размер шрифта для номеров строк
	stepnumber=1, % размер шага между двумя номерами строк
	numbersep=5pt, % как далеко отстоят номера строк от подсвечиваемого кода
	frame=single, % рисовать рамку вокруг кода
	tabsize=2, % размер табуляции по умолчанию равен 4 пробелам
	captionpos=t, % позиция заголовка вверху [t] или внизу [b]
	breaklines=true,
	breakatwhitespace=true, % переносить строки только если есть пробел
	backgroundcolor=\color{white},
	keywordstyle=\color{blue}
}


% для квадратных скобок
\usepackage{array}
\DeclareMathOperator{\rank}{rank}
\makeatletter
\newenvironment{sqcases}{%
	\matrix@check\sqcases\env@sqcases
}{%
	\endarray\right.%
}
\def\env@sqcases{%
	\let\@ifnextchar\new@ifnextchar
	\left\lbrack
	\def\arraystretch{1.2}%
	\array{@{}l@{\quad}l@{}}%
}
\makeatother



\usepackage[justification=centering]{caption} % Настройка подписей float объектов


\usepackage{csvsimple}

\newcommand{\code}[1]{\texttt{#1}}

% для матриц
\makeatletter
\renewcommand*\env@matrix[1][\arraystretch]{%
	\edef\arraystretch{#1}%
	\hskip -\arraycolsep
	\let\@ifnextchar\new@ifnextchar
	\array{*\c@MaxMatrixCols c}}
\makeatother

\usepackage{amsfonts}
\usepackage{graphicx}
\newcommand{\img}[3] {
	\begin{figure}[h!]
		\center{\includegraphics[height=#1]{inc/img/#2}}
		\caption{#3}
		\label{img:#2}
	\end{figure}
}
\newcommand{\boximg}[3] {
	\begin{figure}[h]
		\center{\fbox{\includegraphics[height=#1]{inc/img/#2}}}
		\caption{#3}
		\label{img:#2}
	\end{figure}
}



\usepackage[justification=centering]{caption} % Настройка подписей float объектов


\usepackage{csvsimple}


\setcounter{page}{2}
\usepackage{threeparttable}
\usepackage{dcolumn}
\usepackage{multirow}
\usepackage{longtable}

\usepackage{algorithm}
\usepackage{algpseudocode}

\makeatletter
\makeatother

\algrenewcommand\algorithmicwhile{\textbf{До тех пока}}
\algrenewcommand\algorithmicdo{\textbf{выполнять}}
\algrenewcommand\algorithmicrepeat{\textbf{Повторять}}
\algrenewcommand\algorithmicuntil{\textbf{Пока выполняется}}
\algrenewcommand\algorithmicend{\textbf{Конец}}
\algrenewcommand\algorithmicif{\textbf{Если}}
\algrenewcommand\algorithmicelse{\textbf{иначе}}
\algrenewcommand\algorithmicthen{\textbf{тогда}}
\algrenewcommand\algorithmicfor{\textbf{Цикл}}
\algrenewcommand\algorithmicforall{\textbf{Выполнить для всех}}
\algrenewcommand\algorithmicfunction{\textbf{Функция}}
\algrenewcommand\algorithmicprocedure{\textbf{Процедура}}
\algrenewcommand\algorithmicloop{\textbf{Зациклить}}
\algrenewcommand\algorithmicrequire{\textbf{Условия:}}
\algrenewcommand\algorithmicensure{\textbf{Обеспечивающие условия:}}
\algrenewcommand\algorithmicreturn{\textbf{Возвратить}}
\algrenewtext{EndWhile}{\textbf{Конец цикла}}
\algrenewtext{EndLoop}{\textbf{Конец зацикливания}}
\algrenewtext{EndFor}{\textbf{Конец цикла}}
\algrenewtext{EndFunction}{\textbf{Конец функции}}
\algrenewtext{EndProcedure}{\textbf{Конец процедуры}}
\algrenewtext{EndIf}{\textbf{Конец условия}}
\algrenewtext{EndFor}{\textbf{Конец цикла}}
\algrenewtext{BeginAlgorithm}{\textbf{Начало алгоритма}}
\algrenewtext{EndAlgorithm}{\textbf{Конец алгоритма}}
\algrenewtext{BeginBlock}{\textbf{Начало блока. }}
\algrenewtext{EndBlock}{\textbf{Конец блока}}
\algrenewtext{ElsIf}{\textbf{иначе если }}
\floatname{algorithm}{Алгоритм}

\bibliographystyle{utf8gost705u.bst}
\usepackage[backend=biber,
sorting=none,
]{biblatex} 

\usepackage{tasks}