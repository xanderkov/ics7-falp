\chapter{Практические задания}

\begin{enumerate}[wide=0pt]

	\item \textit{Написать функцию, которая принимает целое число и возвращает первое
	четное число, не меньшее аргумента.}

		\begin{lstlisting}
			(
			defun calc_hyp(a b) (
				sqrt (
						+ 
							(* a a)
							(* b b) 
					)
			)
			)

		\end{lstlisting}


	\item  \textit{Написать функцию, которая принимает число и возвращает число
	того же знака, но с модулем на 1 больше модуля аргумента.}

	\item  \textit{Написать функцию, которая принимает два числа и возвращает спи-
	сок из этих чисел, расположенный по возрастанию.}

	\item  \textit{Написать функцию, которая принимает три числа и возвращает Т
	только тогда, когда первое число расположено между вторым и третьим.}

	\item  \textit{Каков результат вычисления следующих выражений?}
	
		\begin{enumerate}[label=\arabic*)]
			\item (and 'fee 'fie 'foe)
			\item (or nil 'fie 'foe)
			\item (and (equal 'abc 'abc) 'yes)
			\item (or 'fee 'fie 'foe)
			\item (and nil 'fie 'foe)
			\item (or (equal 'abc 'abc) 'yes)
		\end{enumerate}

	\item  \textit{Написать предикат, который принимает два числа-аргумента и 
	возвращает Т, если первое число не меньше второго.}

	\item  \textit{Какой из следующих двух вариантов предиката ошибочен и почему?}
	
		\begin{enumerate}[label=\arabic*)]
			\item (defun pred1 (x))
			\item (and (numberp x) (plusp x)))
			\item (defun pred2 (x))
			\item (and (plusp x)(numberp x)))
		\end{enumerate}

	\item  \textit{Решить задачу 4, используя для ее решения конструкции IF, COND,
	AND/OR.}

	\item  \textit{Переписать функцию how-alike, приведенную в лекции и использующую
	COND, используя только конструкции IF, AND/OR.
	AND/OR:}



\end{enumerate}