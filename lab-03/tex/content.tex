\chapter{Практические задания}

\begin{enumerate}[wide=0pt]

	\item \textit{Написать функцию, которая принимает целое число и возвращает первое
	четное число, не меньшее аргумента.}

		\begin{lstlisting}
			(defun make-even (x)
				(
					if (evenp x)
					x
					(if (> x 0)
						(- x 1)
						(+ x 1)
					)
				)
			)
		\end{lstlisting}


	\item  \textit{Написать функцию, которая принимает число и возвращает число
	того же знака, но с модулем на 1 больше модуля аргумента.}

	\begin{lstlisting}
		(defun calc-plus-one (x)
			(
				if (< x 0)
					(- x 1)
					(+ x 1)
				
			)
		)
	\end{lstlisting}

	\item  \textit{Написать функцию, которая принимает два числа и возвращает спи-
	сок из этих чисел, расположенный по возрастанию.}

	\begin{lstlisting}
		(defun make-two-list (a b)
			(
				if (< a b)
					(list a b)
					(list b a)
				
			)
		)
	\end{lstlisting}

	\item  \textit{Написать функцию, которая принимает три числа и возвращает Т
	только тогда, когда первое число расположено между вторым и третьим.}

	\begin{lstlisting}
		(defun is-first-between (first second third)
			(
				if (or 
						(and (> first second) (< first third))
						(and (> first third) (< first second))
				   )
					t
					nil
				
			)
		)
	\end{lstlisting}

	\item  \textit{Каков результат вычисления следующих выражений?}
	
		\begin{enumerate}[label=\arabic*)]
			\item (and 'fee 'fie 'foe) --- foe
			\item (or nil 'fie 'foe) --- fie
			\item (and (equal 'abc 'abc) 'yes) --- yes
			\item (or 'fee 'fie 'foe) --- fee
			\item (and nil 'fie 'foe) --- Nil
			\item (or (equal 'abc 'abc) 'yes) --- T
		\end{enumerate}

	\item  \textit{Написать предикат, который принимает два числа-аргумента и 
	возвращает Т, если первое число не меньше второго.}

	\begin{lstlisting}
		(defun not-less (x y)
			(
				>= x y
			)
		)
	\end{lstlisting}

	\item  \textit{Какой из следующих двух вариантов предиката ошибочен и почему?}
	
		\begin{enumerate}[label=\arabic*)]
			\item (defun pred1 (x) (and (numberp x) (plusp x)))
			\item (defun pred2 (x) (and (plusp x)(numberp x)))
		\end{enumerate}

		Ответ: ошибочен вариант 2. Так как and вычисляет аргументы до тех пор,
		пока не будет ясно, какой ответ надо вернуть. pred2 при первой проверке
		вернет NIL и завршит работы не вызывая plusp.

	\item  \textit{Решить задачу 4, используя для ее решения конструкции IF, COND,
	AND/OR.}
	
	\begin{lstlisting}
	(defun x-from-y-to-z (x y z)
	(if (< y x)
		(if (< x z)
			T Nil)
		Nil))
	\end{lstlisting}

	\begin{lstlisting}
	(defun x-from-y-to-z (x y z)
		(cond ((< y x) (cond ((< x z) T) (T Nil))) (T Nil)))
	\end{lstlisting}

	\begin{lstlisting}
	(defun x-from-y-to-z (x y z)
		(and (< y x) (< x z)))
	\end{lstlisting}


	\item  \textit{Переписать функцию how-alike, приведенную в лекции и использующую
	COND, используя только конструкции IF, AND/OR.
	AND/OR:}

	\begin{lstlisting}
	(defun how_alike (x y)
    (cond ((or (= x y) (equal x y)) 'the_same)
        ((and (oddp x) (oddp y)) 'both_odd)
        ((and (evenp x) (evenp y)) 'both_even)
        (T 'difference)
	))
	\end{lstlisting}


	\begin{lstlisting}
		(defun how_alike (x y)
		(if (if (= x y)
				(equal x y))
			'the_same
			(if (if (oddp x)
				(oddp y))
			'both_odd
			(if (if (evenp x)
				(evenp y))
			'both_even
			'difference
		))))
	\end{lstlisting}


	\begin{lstlisting}
		(defun how_alike (x y)
		(or
			(and (= x y) (equal x y) 'the_same)
			(and (oddp x) (oddp y) 'both_odd)
			(and (evenp x) (evenp y) 'both_even)
			'difference
		))
	\end{lstlisting}



\end{enumerate}