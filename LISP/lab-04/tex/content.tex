\chapter{Практические задания}

\begin{enumerate}[wide=0pt]

	\item \textit{Чем принципиально отличаются функции cons, list, append?}

		\begin{lstlisting}
			(setf lst1 '(a b))
			(setf lst2 '(c d))
		\end{lstlisting}

		\textit{Каковы результаты вычисления следующих выражений?}
		\begin{lstlisting}
			(cons lstl lst2)
			(list lst1 lst2)
			(append lst1 lst2)
		\end{lstlisting}

		\begin{itemize}
			\item cons объединяет значения своих аргументов в точечную пару. Если вторым
			аргументом будет передан список, то в результате получится список, в
			котором второй аргумент будет добавлен в начало: ((A B)C D)

			\item list составляет из своих аргументов список: ((A B) (C D))
			\item append создает копию всех аргументов, кроме последнего, т. е списковые
			ячейки. Связываются последними указателями. Результирующее значе-
			ние: (A B C D)
		\end{itemize}


	\item  \textit{Каковы результаты вычисления следующих выражений, и почему?}

	\begin{lstlisting}
		(reverse '(a b c))        --> Nil
		(reverse '(a b (c (d))))  --> Nil
		(reverse '(a))            --> (A)
		(last '(a b c))           --> (C) 
		(last '(a))               --> (A)
		(last '((a b c)))         --> ((a b c))
		(reverse ())              --> Nil
		(reverse '((a b c)))      -->  ((A B C))
		(last '(a b (c)))         --> ((c))
		(last ())                 --> Nil
	\end{lstlisting}

	\item  \textit{Написать, по крайней мере, два варианта функции, 
	которая возвращает
	последний элемент своего списка-аргумента.}

	\begin{lstlisting}
		(defun get-last (lst)
			(
				car (last lst)
			)
		)
	\end{lstlisting}

	\begin{lstlisting}
		(defun get-last-reverse (lst)
			(
				car (reverse lst)
			)
		)
	


	\end{lstlisting}

	\item  \textit{Написать, по крайней мере, два варианта функции, 
	которая возвращает
	свой список аргумент без последнего элемента.}

	\begin{lstlisting}
		(defun get-last (lst)
			(
				nbutlast lst 1
			)
		)
	\end{lstlisting}

	\begin{lstlisting}
		(defun get-without-last-reverse (lst)
			(
				reverse (cdr (reverse lst))
			)
		)
	
	
	\end{lstlisting}

	\item  \textit{Напишите функцию swap-first-last, 
	которая переставляет в списке-
	аргументе первый и последний элементы.}
	
	\begin{lstlisting}
		(defun swap-first-last (lst)
			(
				nconc 
				(last lst)
				(reverse 
					(cdr 
						(reverse (cdr lst)))
				)
				(list (car lst))
			)
		)
	\end{lstlisting}

	\item  \textit{Написать простой вариант игры в кости, в котором бросаются две
	правильные кости. Если сумма выпавших очков равна 7 или 11 —
	выигрыш, если выпало (1,1) или (6,6) — игрок имеет право снова
	бросить кости, во всех остальных случаях ход переходит ко второму
	игроку, но запоминается сумма выпавших очков. Если второй игрок не
	выигрывает абсолютно, то выигрывает тот игрок, у которого большеочков. Результат игры и значения выпавших костей выводить на экран с
	помощью функции print.}

	\begin{lstlisting}
		(defvar first_player)
		(defvar second_player)
		
		
		(defun bones_throw ()
		
			(print "Enter first bone: ")
			(setq bone1 (read))
			(print "Enter second bone: ")
			(setq bone2 (read))
			(setq ret (list bone1 bone2))
			ret
		
		)
		
		
		(defun check-easy-win (lst)
			(
				or ( = (+ (car lst) (cadr lst)) 7)
				   ( = (+ (car lst) (cadr lst)) 11)
			)
		)
		
		(defun pass_check (lst)
			(
				or
				(   
					and (= (car lst) 1) (= (cadr lst) 1)
				)
				(   
					and (= (car lst) 6) (= (cadr lst) 6)
				)
			)
		)
		
		
		(defun play-game-second-player ()
			(print "Second player throw: ") 
			(setq second_player (bones_throw))
			(print second_player)
		
			(
				cond 
				(
					(check-easy-win second_player)
					(print "Second player wins") 
				)
				(
					(pass_check second_player)
					(play-game-second-player)
				)
				(
					T
					(
						cond
						(
							(
								>
								(+ (car first_player) (cadr first_player))
								(+ (car second_player) (cadr second_player))
							)
							(print "First player wins") 
						)
						(
							(
								<
								(+ (car first_player) (cadr first_player))
								(+ (car second_player) (cadr second_player))
							)
							(print "Second player wins") 
						)
						(
							T
							(print "Draw in the game")
						)
					)
				)
			)
		)
		
		
		(defun play-game-first-player ()
		
			(print "First player throws: ")
			(setq first_player (bones_throw))
			(print first_player)
		
			(
				cond
				(
					(check-easy-win first_player)
					(print "First player wins")
				)
				(
					(pass_check first_player)
					(play-game-first-player)
				)
				(
					T
					(play-game-second-player)
				)
			)
		
		)
		
		
		(play-game-first-player)
	\end{lstlisting}

	\item  \textit{Написать функцию, которая по своему 
	списку-аргументу lst определяет
	является ли он палиндромом (то есть равны ли lst и (reverse lst)).}
	\begin{lstlisting}
		(defun get-without-last-reverse (lst)
			(reverse (cdr (reverse lst)))
		)



		(defun st_check (lst)
			(cond
				(
					(> (length lst) 1)
					(and
						(eq (car lst) (car (reverse lst)))
						(st_check (cdr (get-without-last-reverse lst)))
					)
				)
				(T T)
			)
		)


		(defun palindrom_check (lst)
			(st_check lst)
		)
	\end{lstlisting}

	\item  \textit{Напишите свои необходимые функции, 
	которые обрабатывают таблицу из
	4-х точечных пар:
	(страна . столица), и возвращают по стране - столицу, а по столице —
	страну.}
	
	\begin{lstlisting}
		(defun countries_capitals (lst name)
			(
				cond 
				(
					(assoc name lst)
					(cdr (assoc name lst))
				)
				(
					(rassoc name lst)
					(car (rassoc name lst))
				)
				(T Nil)
			)
		)
	\end{lstlisting}



	\item  \textit{Напишите функцию, которая умножает на заданное число-аргумент
	первый числовой элемент списка из заданного 3-х элементного списка-
	аргумента, когда
	a) все элементы списка --- числа,
	6) элементы списка -- любые объекты.}

	\begin{lstlisting}
		(defun mult_el_a (n lst)
			(
				cond 
				(
					(and
						(and
							(numberp (car lst))
							(and (numberp (cadr lst)) (numberp (caddr lst)))
						)
						(numberp n)
					)
					(* (car lst) n)
				)
				(T Nil)
			)
		)
	\end{lstlisting}


	\begin{lstlisting}
		(defun mult_el_b (n lst)
			(cond 
				(
					(and (numberp (car lst)) (numberp n))
					(* (car lst) n)
				)
				(T Nil)
			)
		)
	\end{lstlisting}

\end{enumerate}